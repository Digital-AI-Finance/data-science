\documentclass[8pt,aspectratio=169]{beamer}
\usetheme{Madrid}
\usepackage{graphicx}
\usepackage{booktabs}
\usepackage{amsmath}

\definecolor{mlpurple}{RGB}{51,51,178}
\definecolor{mllavender}{RGB}{173,173,224}
\definecolor{mllavender3}{RGB}{204,204,235}

\setbeamercolor{palette primary}{bg=mllavender3,fg=mlpurple}
\setbeamercolor{structure}{fg=mlpurple}
\setbeamercolor{frametitle}{fg=mlpurple,bg=mllavender3}

\setbeamertemplate{navigation symbols}{}
\setbeamersize{text margin left=5mm,text margin right=5mm}

\newcommand{\bottomnote}[1]{\vfill\footnotesize\textbf{#1}}


\title{Lesson 13: Descriptive Statistics}
\subtitle{Data Science with Python -- BSc Course}
\date{45 Minutes}

\begin{document}

\begin{frame}[plain]
\titlepage
\end{frame}

\begin{frame}[t]{Learning Objectives}
\textbf{After this lesson, you will be able to:}
\begin{itemize}
\item Calculate mean, median, mode
\item Measure dispersion (std, variance, range)
\item Interpret quartiles and percentiles
\item Analyze skewness and kurtosis
\end{itemize}
\bottomnote{Finance application: Statistical analysis of market data}
\end{frame}


\begin{frame}[t]{Central Tendency}
\begin{center}
\includegraphics[width=0.85\textwidth]{01_central_tendency/chart.pdf}
\end{center}
\bottomnote{Statistical foundation for data-driven decisions}
\end{frame}


\begin{frame}[t]{Dispersion}
\begin{center}
\includegraphics[width=0.85\textwidth]{02_dispersion/chart.pdf}
\end{center}
\bottomnote{Statistical foundation for data-driven decisions}
\end{frame}


\begin{frame}[t]{Quartiles}
\begin{center}
\includegraphics[width=0.85\textwidth]{03_quartiles/chart.pdf}
\end{center}
\bottomnote{Statistical foundation for data-driven decisions}
\end{frame}


\begin{frame}[t]{Skewness}
\begin{center}
\includegraphics[width=0.85\textwidth]{04_skewness/chart.pdf}
\end{center}
\bottomnote{Statistical foundation for data-driven decisions}
\end{frame}


\begin{frame}[t]{Kurtosis}
\begin{center}
\includegraphics[width=0.85\textwidth]{05_kurtosis/chart.pdf}
\end{center}
\bottomnote{Statistical foundation for data-driven decisions}
\end{frame}


\begin{frame}[t]{Summary Table}
\begin{center}
\includegraphics[width=0.85\textwidth]{06_summary_table/chart.pdf}
\end{center}
\bottomnote{Statistical foundation for data-driven decisions}
\end{frame}


\begin{frame}[t]{Finance Stats}
\begin{center}
\includegraphics[width=0.85\textwidth]{07_finance_stats/chart.pdf}
\end{center}
\bottomnote{Statistical foundation for data-driven decisions}
\end{frame}


\begin{frame}[t]{Comparison}
\begin{center}
\includegraphics[width=0.85\textwidth]{08_comparison/chart.pdf}
\end{center}
\bottomnote{Statistical foundation for data-driven decisions}
\end{frame}


\begin{frame}[t]{Lesson Summary}
\textbf{Key Takeaways:}
\begin{itemize}
\item Calculate mean, median, mode
\item Measure dispersion (std, variance, range)
\item Interpret quartiles and percentiles
\item Analyze skewness and kurtosis
\end{itemize}
\bottomnote{Statistics + Visualization = Data Science foundation}
\end{frame}

\end{document}
