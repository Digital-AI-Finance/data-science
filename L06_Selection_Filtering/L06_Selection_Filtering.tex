\documentclass[8pt,aspectratio=169]{beamer}
\usetheme{Madrid}
\usepackage{graphicx}
\usepackage{booktabs}
\usepackage{adjustbox}
\usepackage{multicol}
\usepackage{amsmath}
\usepackage{amssymb}

% Color definitions
\definecolor{mlblue}{RGB}{0,102,204}
\definecolor{mlpurple}{RGB}{51,51,178}
\definecolor{mllavender}{RGB}{173,173,224}
\definecolor{mllavender2}{RGB}{193,193,232}
\definecolor{mllavender3}{RGB}{204,204,235}
\definecolor{mllavender4}{RGB}{214,214,239}
\definecolor{mlorange}{RGB}{255, 127, 14}
\definecolor{mlgreen}{RGB}{44, 160, 44}
\definecolor{mlred}{RGB}{214, 39, 40}

\setbeamercolor{palette primary}{bg=mllavender3,fg=mlpurple}
\setbeamercolor{palette secondary}{bg=mllavender2,fg=mlpurple}
\setbeamercolor{palette tertiary}{bg=mllavender,fg=white}
\setbeamercolor{palette quaternary}{bg=mlpurple,fg=white}
\setbeamercolor{structure}{fg=mlpurple}
\setbeamercolor{frametitle}{fg=mlpurple,bg=mllavender3}
\setbeamercolor{block title}{bg=mllavender2,fg=mlpurple}
\setbeamercolor{block body}{bg=mllavender4,fg=black}

\setbeamertemplate{navigation symbols}{}
\setbeamertemplate{itemize items}[circle]
\setbeamersize{text margin left=5mm,text margin right=5mm}

\newcommand{\bottomnote}[1]{%
\vfill\vspace{-2mm}
\textcolor{mllavender2}{\rule{\textwidth}{0.4pt}}
\vspace{1mm}\footnotesize\textbf{#1}}


\title{Lesson 06: Selection and Filtering}
\subtitle{Data Science with Python -- BSc Course}
\author{Data Science Program}
\date{45 Minutes}

\begin{document}

\begin{frame}[plain]
\titlepage
\end{frame}

\begin{frame}[t]{Learning Objectives}
\textbf{After this lesson, you will be able to:}
\begin{itemize}
\item Select columns using bracket and dot notation
\item Access rows with iloc (position) and loc (label)
\item Filter data using boolean conditions
\item Combine multiple conditions with \& and |
\end{itemize}

\vspace{1em}
\textbf{Finance Application:} Screen stocks by price, volume, and other criteria.

\bottomnote{Selection and filtering extract relevant data for analysis}
\end{frame}

\begin{frame}[t]{Column Selection}
\begin{center}
\includegraphics[width=0.85\textwidth]{01_column_selection/chart.pdf}
\end{center}

\bottomnote{Single brackets return Series; double brackets return DataFrame}
\end{frame}

\begin{frame}[t]{iloc vs loc}
\begin{center}
\includegraphics[width=0.85\textwidth]{02_iloc_vs_loc/chart.pdf}
\end{center}

\bottomnote{iloc: integer position; loc: label-based}
\end{frame}

\begin{frame}[t]{Boolean Masking}
\begin{center}
\includegraphics[width=0.85\textwidth]{03_boolean_mask/chart.pdf}
\end{center}

\bottomnote{Boolean mask selects rows where condition is True}
\end{frame}

\begin{frame}[t]{Conditional Filtering}
\begin{center}
\includegraphics[width=0.85\textwidth]{04_conditional_filtering/chart.pdf}
\end{center}

\bottomnote{Filtering creates a new DataFrame -- original unchanged}
\end{frame}

\begin{frame}[t]{Multiple Conditions}
\begin{center}
\includegraphics[width=0.85\textwidth]{05_multiple_conditions/chart.pdf}
\end{center}

\bottomnote{Use parentheses around each condition!}
\end{frame}

\begin{frame}[t]{Chained Filtering}
\begin{center}
\includegraphics[width=0.85\textwidth]{06_chained_filtering/chart.pdf}
\end{center}

\bottomnote{query() method is more readable for complex filters}
\end{frame}

\begin{frame}[t]{Selection Methods Comparison}
\begin{center}
\includegraphics[width=0.85\textwidth]{07_selection_comparison/chart.pdf}
\end{center}

\bottomnote{Choose method based on what you need to select}
\end{frame}

\begin{frame}[t]{Stock Screening Workflow}
\begin{center}
\includegraphics[width=0.85\textwidth]{08_stock_screening/chart.pdf}
\end{center}

\bottomnote{Stock screening = loading + filtering + selecting + sorting}
\end{frame}

\begin{frame}[t]{Hands-on Exercise (25 min)}
\textbf{Build a stock screener:}

\begin{enumerate}
\item Load stock data from CSV

\item Select only AAPL and MSFT columns

\item Filter rows where AAPL > 185

\item Filter rows where AAPL > 185 AND MSFT > 375

\item Use query() for the same filter

\item Select first 10 trading days using iloc

\item Sort by AAPL price descending
\end{enumerate}

\bottomnote{Stock screeners are fundamental tools in finance}
\end{frame}

\begin{frame}[t]{Lesson Summary}
\textbf{Key Takeaways:}
\begin{itemize}
\item df["col"] selects single column as Series
\item iloc uses integer positions; loc uses labels
\item Boolean conditions create True/False masks
\item Combine conditions with \& (and) and | (or)
\item query() is cleaner for complex filters
\end{itemize}

\vspace{1em}
\textbf{Next Lesson:} Missing Data and Cleaning

\bottomnote{Week 1 complete! You can now load, explore, and filter data}
\end{frame}

\end{document}
