\documentclass[8pt,aspectratio=169]{beamer}
\usetheme{Madrid}
\usepackage{graphicx}
\usepackage{booktabs}
\usepackage{amsmath}

\definecolor{mlpurple}{RGB}{51,51,178}
\definecolor{mllavender}{RGB}{173,173,224}
\definecolor{mllavender3}{RGB}{204,204,235}

\setbeamercolor{palette primary}{bg=mllavender3,fg=mlpurple}
\setbeamercolor{structure}{fg=mlpurple}
\setbeamercolor{frametitle}{fg=mlpurple,bg=mllavender3}

\setbeamertemplate{navigation symbols}{}
\setbeamersize{text margin left=5mm,text margin right=5mm}

\newcommand{\bottomnote}[1]{\vfill\footnotesize\textbf{#1}}


\title{Lesson 34: MLPs and Activations}
\subtitle{Data Science with Python -- BSc Course}
\date{45 Minutes}

\begin{document}

\begin{frame}[plain]
\titlepage
\end{frame}

\begin{frame}[t]{Learning Objectives}
\textbf{After this lesson, you will be able to:}
\begin{itemize}
\item Design MLP architectures
\item Choose activation functions
\item Build models with Keras
\item Apply to non-linear problems
\end{itemize}
\bottomnote{Building towards your final project}
\end{frame}


\begin{frame}[t]{Mlp Architecture}
\begin{center}
\includegraphics[width=0.85\textwidth]{01_mlp_architecture/chart.pdf}
\end{center}
\end{frame}


\begin{frame}[t]{Relu Activation}
\begin{center}
\includegraphics[width=0.85\textwidth]{02_relu_activation/chart.pdf}
\end{center}
\end{frame}


\begin{frame}[t]{Sigmoid Softmax}
\begin{center}
\includegraphics[width=0.85\textwidth]{03_sigmoid_softmax/chart.pdf}
\end{center}
\end{frame}


\begin{frame}[t]{Keras Sequential}
\begin{center}
\includegraphics[width=0.85\textwidth]{04_keras_sequential/chart.pdf}
\end{center}
\end{frame}


\begin{frame}[t]{Hidden Layers}
\begin{center}
\includegraphics[width=0.85\textwidth]{05_hidden_layers/chart.pdf}
\end{center}
\end{frame}


\begin{frame}[t]{Parameter Counting}
\begin{center}
\includegraphics[width=0.85\textwidth]{06_parameter_counting/chart.pdf}
\end{center}
\end{frame}


\begin{frame}[t]{Universal Approximation}
\begin{center}
\includegraphics[width=0.85\textwidth]{07_universal_approximation/chart.pdf}
\end{center}
\end{frame}


\begin{frame}[t]{Market Regimes}
\begin{center}
\includegraphics[width=0.85\textwidth]{08_market_regimes/chart.pdf}
\end{center}
\end{frame}


\begin{frame}[t]{Lesson Summary}
\textbf{Key Takeaways:}
\begin{itemize}
\item Design MLP architectures
\item Choose activation functions
\item Build models with Keras
\item Apply to non-linear problems
\end{itemize}
\bottomnote{Apply these skills in your final project}
\end{frame}

\end{document}
