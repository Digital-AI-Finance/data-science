\documentclass[8pt,aspectratio=169]{beamer}
\usetheme{Madrid}
\usepackage{graphicx}
\usepackage{booktabs}
\usepackage{amsmath}

\definecolor{mlpurple}{RGB}{51,51,178}
\definecolor{mllavender}{RGB}{173,173,224}
\definecolor{mllavender2}{RGB}{193,193,232}
\definecolor{mllavender3}{RGB}{204,204,235}
\definecolor{mllavender4}{RGB}{214,214,239}
\definecolor{mlorange}{RGB}{255, 127, 14}
\definecolor{mlgreen}{RGB}{44, 160, 44}

\setbeamercolor{palette primary}{bg=mllavender3,fg=mlpurple}
\setbeamercolor{structure}{fg=mlpurple}
\setbeamercolor{frametitle}{fg=mlpurple,bg=mllavender3}

\setbeamertemplate{navigation symbols}{}
\setbeamersize{text margin left=5mm,text margin right=5mm}

\newcommand{\bottomnote}[1]{\vfill\footnotesize\textbf{#1}}


\title{Lesson 12: Time Series Basics}
\subtitle{Data Science with Python -- BSc Course}
\date{45 Minutes}

\begin{document}

\begin{frame}[plain]
\titlepage
\end{frame}


\begin{frame}[t]{Learning Objectives}
\textbf{After this lesson, you will be able to:}
\begin{itemize}
\item DateTime index
\item Resampling (daily to monthly)
\item Rolling windows
\item shift() and pct\_change()
\item Time series patterns in finance
\end{itemize}
\bottomnote{Finance application: Stock data processing and analysis}
\end{frame}


\begin{frame}[t]{Basic Time Series}
\begin{center}
\includegraphics[width=0.75\textwidth]{01a_basic_timeseries/chart.pdf}
\end{center}
\bottomnote{Plotting time-indexed data}
\end{frame}


\begin{frame}[t]{Monthly Aggregation}
\begin{center}
\includegraphics[width=0.75\textwidth]{01b_monthly_aggregation/chart.pdf}
\end{center}
\bottomnote{Summarizing daily data into monthly periods}
\end{frame}


\begin{frame}[t]{Date Slicing}
\begin{center}
\includegraphics[width=0.75\textwidth]{01c_date_slicing/chart.pdf}
\end{center}
\bottomnote{Selecting data by date range}
\end{frame}


\begin{frame}[t]{Day of Week Patterns}
\begin{center}
\includegraphics[width=0.75\textwidth]{01d_day_of_week/chart.pdf}
\end{center}
\bottomnote{Analyzing weekday effects}
\end{frame}


\begin{frame}[t]{Datetime Parsing}
\begin{center}
\includegraphics[width=0.75\textwidth]{02_datetime_parsing/chart.pdf}
\end{center}
\bottomnote{Converting strings to datetime objects}
\end{frame}


\begin{frame}[t]{Daily Data}
\begin{center}
\includegraphics[width=0.75\textwidth]{03a_daily_data/chart.pdf}
\end{center}
\bottomnote{Working with daily frequency}
\end{frame}


\begin{frame}[t]{Weekly Resampling}
\begin{center}
\includegraphics[width=0.75\textwidth]{03b_weekly_resampling/chart.pdf}
\end{center}
\bottomnote{Aggregating to weekly frequency}
\end{frame}


\begin{frame}[t]{Monthly OHLC}
\begin{center}
\includegraphics[width=0.75\textwidth]{03c_monthly_ohlc/chart.pdf}
\end{center}
\bottomnote{Open-High-Low-Close monthly bars}
\end{frame}


\begin{frame}[t]{Quarterly Resampling}
\begin{center}
\includegraphics[width=0.75\textwidth]{03d_quarterly/chart.pdf}
\end{center}
\bottomnote{Aggregating to quarterly frequency}
\end{frame}


\begin{frame}[t]{Moving Averages}
\begin{center}
\includegraphics[width=0.75\textwidth]{04a_moving_averages/chart.pdf}
\end{center}
\bottomnote{Smoothing with rolling mean}
\end{frame}


\begin{frame}[t]{Rolling Volatility}
\begin{center}
\includegraphics[width=0.75\textwidth]{04b_rolling_volatility/chart.pdf}
\end{center}
\bottomnote{Measuring time-varying risk}
\end{frame}


\begin{frame}[t]{Bollinger Bands}
\begin{center}
\includegraphics[width=0.75\textwidth]{04c_bollinger_bands/chart.pdf}
\end{center}
\bottomnote{Trading bands using rolling statistics}
\end{frame}


\begin{frame}[t]{Rolling Channel}
\begin{center}
\includegraphics[width=0.75\textwidth]{04d_rolling_channel/chart.pdf}
\end{center}
\bottomnote{Price channels with rolling min/max}
\end{frame}


\begin{frame}[t]{Shift Demo}
\begin{center}
\includegraphics[width=0.75\textwidth]{05a_shift_demo/chart.pdf}
\end{center}
\bottomnote{Basic lag operation}
\end{frame}


\begin{frame}[t]{Multiple Lags}
\begin{center}
\includegraphics[width=0.75\textwidth]{05b_multiple_lags/chart.pdf}
\end{center}
\bottomnote{Creating multiple lagged features}
\end{frame}


\begin{frame}[t]{Returns with Shift}
\begin{center}
\includegraphics[width=0.75\textwidth]{05c_returns_shift/chart.pdf}
\end{center}
\bottomnote{Computing returns using shift}
\end{frame}


\begin{frame}[t]{Autocorrelation}
\begin{center}
\includegraphics[width=0.75\textwidth]{05d_autocorrelation/chart.pdf}
\end{center}
\bottomnote{Serial correlation in returns}
\end{frame}


\begin{frame}[t]{Daily Returns}
\begin{center}
\includegraphics[width=0.75\textwidth]{06a_daily_returns/chart.pdf}
\end{center}
\bottomnote{Computing percentage changes}
\end{frame}


\begin{frame}[t]{Periods Comparison}
\begin{center}
\includegraphics[width=0.75\textwidth]{06b_periods_comparison/chart.pdf}
\end{center}
\bottomnote{Different return periods}
\end{frame}


\begin{frame}[t]{Return Distribution}
\begin{center}
\includegraphics[width=0.75\textwidth]{06c_return_distribution/chart.pdf}
\end{center}
\bottomnote{Statistical properties of returns}
\end{frame}


\begin{frame}[t]{Cumulative Returns}
\begin{center}
\includegraphics[width=0.75\textwidth]{06d_cumulative_returns/chart.pdf}
\end{center}
\bottomnote{Growth of investment over time}
\end{frame}


\begin{frame}[t]{Original Trend}
\begin{center}
\includegraphics[width=0.75\textwidth]{07a_original_trend/chart.pdf}
\end{center}
\bottomnote{Identifying the long-term trend}
\end{frame}


\begin{frame}[t]{Seasonal Pattern}
\begin{center}
\includegraphics[width=0.75\textwidth]{07b_seasonal_pattern/chart.pdf}
\end{center}
\bottomnote{Recurring patterns within periods}
\end{frame}


\begin{frame}[t]{Residuals}
\begin{center}
\includegraphics[width=0.75\textwidth]{07c_residuals/chart.pdf}
\end{center}
\bottomnote{Random component after decomposition}
\end{frame}


\begin{frame}[t]{Full Decomposition}
\begin{center}
\includegraphics[width=0.75\textwidth]{07d_decomposition/chart.pdf}
\end{center}
\bottomnote{Trend + Seasonal + Residual}
\end{frame}


\begin{frame}[t]{Mean Reversion}
\begin{center}
\includegraphics[width=0.75\textwidth]{08a_mean_reversion/chart.pdf}
\end{center}
\bottomnote{Price returning to average}
\end{frame}


\begin{frame}[t]{Momentum}
\begin{center}
\includegraphics[width=0.75\textwidth]{08b_momentum/chart.pdf}
\end{center}
\bottomnote{Trend-following patterns}
\end{frame}


\begin{frame}[t]{Volatility Clustering}
\begin{center}
\includegraphics[width=0.75\textwidth]{08c_volatility_clustering/chart.pdf}
\end{center}
\bottomnote{High volatility follows high volatility}
\end{frame}


\begin{frame}[t]{Regime Changes}
\begin{center}
\includegraphics[width=0.75\textwidth]{08d_regime_changes/chart.pdf}
\end{center}
\bottomnote{Different market conditions}
\end{frame}


\begin{frame}[t]{Lesson Summary}
\textbf{Key Takeaways:}
\begin{itemize}
\item DateTime index for time series data
\item Resampling changes data frequency
\item Rolling windows for moving statistics
\item shift() creates lags, pct\_change() computes returns
\item Recognize patterns: trend, seasonality, mean reversion, momentum
\end{itemize}

\vspace{1em}
\textbf{Practice:} Apply these concepts to the stock price dataset.
\end{frame}

\end{document}
