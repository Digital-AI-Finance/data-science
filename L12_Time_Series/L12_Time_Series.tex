\documentclass[8pt,aspectratio=169]{beamer}
\usetheme{Madrid}
\usepackage{graphicx}
\usepackage{booktabs}
\usepackage{amsmath}

\definecolor{mlpurple}{RGB}{51,51,178}
\definecolor{mllavender}{RGB}{173,173,224}
\definecolor{mllavender2}{RGB}{193,193,232}
\definecolor{mllavender3}{RGB}{204,204,235}
\definecolor{mllavender4}{RGB}{214,214,239}
\definecolor{mlorange}{RGB}{255, 127, 14}
\definecolor{mlgreen}{RGB}{44, 160, 44}

\setbeamercolor{palette primary}{bg=mllavender3,fg=mlpurple}
\setbeamercolor{structure}{fg=mlpurple}
\setbeamercolor{frametitle}{fg=mlpurple,bg=mllavender3}

\setbeamertemplate{navigation symbols}{}
\setbeamersize{text margin left=5mm,text margin right=5mm}

\newcommand{\bottomnote}[1]{\vfill\footnotesize\textbf{#1}}


\title{Lesson 12: Time Series Basics}
\subtitle{Data Science with Python -- BSc Course}
\date{45 Minutes}

\begin{document}

\begin{frame}[plain]
\titlepage
\end{frame}


\begin{frame}[t]{Learning Objectives}
\textbf{After this lesson, you will be able to:}
\begin{itemize}
\item DateTime index
\item Resampling (daily to monthly)
\item Rolling windows
\item shift() and pct_change()
\item Time series patterns in finance
\end{itemize}
\bottomnote{Finance application: Stock data processing and analysis}
\end{frame}


\begin{frame}[t]{Time Series}
\begin{center}
\includegraphics[width=0.85\textwidth]{01_time_series/chart.pdf}
\end{center}
\bottomnote{Key concept for financial data analysis}
\end{frame}


\begin{frame}[t]{Datetime Parsing}
\begin{center}
\includegraphics[width=0.85\textwidth]{02_datetime_parsing/chart.pdf}
\end{center}
\bottomnote{Key concept for financial data analysis}
\end{frame}


\begin{frame}[t]{03 Resampling}
\begin{center}
\includegraphics[width=0.85\textwidth]{03_resampling/chart.pdf}
\end{center}
\bottomnote{Key concept for financial data analysis}
\end{frame}


\begin{frame}[t]{04 Rolling Window}
\begin{center}
\includegraphics[width=0.85\textwidth]{04_rolling_window/chart.pdf}
\end{center}
\bottomnote{Key concept for financial data analysis}
\end{frame}


\begin{frame}[t]{05 Shift Lag}
\begin{center}
\includegraphics[width=0.85\textwidth]{05_shift_lag/chart.pdf}
\end{center}
\bottomnote{Key concept for financial data analysis}
\end{frame}


\begin{frame}[t]{06 Pct Change}
\begin{center}
\includegraphics[width=0.85\textwidth]{06_pct_change/chart.pdf}
\end{center}
\bottomnote{Key concept for financial data analysis}
\end{frame}


\begin{frame}[t]{07 Components}
\begin{center}
\includegraphics[width=0.85\textwidth]{07_components/chart.pdf}
\end{center}
\bottomnote{Key concept for financial data analysis}
\end{frame}


\begin{frame}[t]{08 Patterns}
\begin{center}
\includegraphics[width=0.85\textwidth]{08_patterns/chart.pdf}
\end{center}
\bottomnote{Key concept for financial data analysis}
\end{frame}


\begin{frame}[t]{Lesson Summary}
\textbf{Key Takeaways:}
\begin{itemize}
\item DateTime index
\item Resampling (daily to monthly)
\item Rolling windows
\item shift() and pct_change()
\item Time series patterns in finance
\end{itemize}

\vspace{1em}
\textbf{Practice:} Apply these concepts to the stock price dataset.
\end{frame}

\end{document}
