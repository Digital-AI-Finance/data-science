\documentclass[8pt,aspectratio=169]{beamer}
\usetheme{Madrid}
\usepackage{graphicx}
\usepackage{booktabs}
\usepackage{amsmath}

\definecolor{mlpurple}{RGB}{51,51,178}
\definecolor{mllavender}{RGB}{173,173,224}
\definecolor{mllavender3}{RGB}{204,204,235}

\setbeamercolor{palette primary}{bg=mllavender3,fg=mlpurple}
\setbeamercolor{structure}{fg=mlpurple}
\setbeamercolor{frametitle}{fg=mlpurple,bg=mllavender3}

\setbeamertemplate{navigation symbols}{}
\setbeamersize{text margin left=5mm,text margin right=5mm}

\newcommand{\bottomnote}[1]{\vfill\footnotesize\textbf{#1}}


\title{Lesson 26: Decision Trees}
\subtitle{Data Science with Python -- BSc Course}
\date{45 Minutes}

\begin{document}

\begin{frame}[plain]
\titlepage
\end{frame}

\begin{frame}[t]{Learning Objectives}
\textbf{After this lesson, you will be able to:}
\begin{itemize}
\item Build decision tree classifiers
\item Understand splitting criteria
\item Apply Random Forest ensemble
\item Interpret feature importance
\end{itemize}
\bottomnote{Building towards your final project}
\end{frame}


\begin{frame}[t]{Tree Structure}
\begin{center}
\includegraphics[width=0.85\textwidth]{01_tree_structure/chart.pdf}
\end{center}
\end{frame}


\begin{frame}[t]{Gini Entropy}
\begin{center}
\includegraphics[width=0.85\textwidth]{02_gini_entropy/chart.pdf}
\end{center}
\end{frame}


\begin{frame}[t]{Tree Sklearn}
\begin{center}
\includegraphics[width=0.85\textwidth]{03_tree_sklearn/chart.pdf}
\end{center}
\end{frame}


\begin{frame}[t]{Overfitting}
\begin{center}
\includegraphics[width=0.85\textwidth]{04_overfitting/chart.pdf}
\end{center}
\end{frame}


\begin{frame}[t]{Random Forest}
\begin{center}
\includegraphics[width=0.85\textwidth]{05_random_forest/chart.pdf}
\end{center}
\end{frame}


\begin{frame}[t]{Feature Importance}
\begin{center}
\includegraphics[width=0.85\textwidth]{06_feature_importance/chart.pdf}
\end{center}
\end{frame}


\begin{frame}[t]{Tree Visualization}
\begin{center}
\includegraphics[width=0.85\textwidth]{07_tree_visualization/chart.pdf}
\end{center}
\end{frame}


\begin{frame}[t]{Finance Trees}
\begin{center}
\includegraphics[width=0.85\textwidth]{08_finance_trees/chart.pdf}
\end{center}
\end{frame}


\begin{frame}[t]{Lesson Summary}
\textbf{Key Takeaways:}
\begin{itemize}
\item Build decision tree classifiers
\item Understand splitting criteria
\item Apply Random Forest ensemble
\item Interpret feature importance
\end{itemize}
\bottomnote{Apply these skills in your final project}
\end{frame}

\end{document}
