\documentclass[8pt,aspectratio=169]{beamer}
\usetheme{Madrid}
\usepackage{graphicx}
\usepackage{booktabs}
\usepackage{adjustbox}
\usepackage{multicol}
\usepackage{amsmath}
\usepackage{amssymb}

% Color definitions
\definecolor{mlblue}{RGB}{0,102,204}
\definecolor{mlpurple}{RGB}{51,51,178}
\definecolor{mllavender}{RGB}{173,173,224}
\definecolor{mllavender2}{RGB}{193,193,232}
\definecolor{mllavender3}{RGB}{204,204,235}
\definecolor{mllavender4}{RGB}{214,214,239}
\definecolor{mlorange}{RGB}{255, 127, 14}
\definecolor{mlgreen}{RGB}{44, 160, 44}
\definecolor{mlred}{RGB}{214, 39, 40}

\setbeamercolor{palette primary}{bg=mllavender3,fg=mlpurple}
\setbeamercolor{palette secondary}{bg=mllavender2,fg=mlpurple}
\setbeamercolor{palette tertiary}{bg=mllavender,fg=white}
\setbeamercolor{palette quaternary}{bg=mlpurple,fg=white}
\setbeamercolor{structure}{fg=mlpurple}
\setbeamercolor{frametitle}{fg=mlpurple,bg=mllavender3}
\setbeamercolor{block title}{bg=mllavender2,fg=mlpurple}
\setbeamercolor{block body}{bg=mllavender4,fg=black}

\setbeamertemplate{navigation symbols}{}
\setbeamertemplate{itemize items}[circle]
\setbeamersize{text margin left=5mm,text margin right=5mm}

\newcommand{\bottomnote}[1]{%
\vfill\vspace{-2mm}
\textcolor{mllavender2}{\rule{\textwidth}{0.4pt}}
\vspace{1mm}\footnotesize\textbf{#1}}

\title{Lesson 01: Python Setup}
\subtitle{Data Science with Python -- BSc Course}
\author{Data Science Program}
\date{45 Minutes}

\begin{document}

% Title slide
\begin{frame}[plain]
\titlepage
\end{frame}

% Learning Objectives
\begin{frame}[t]{Learning Objectives}
\textbf{After this lesson, you will be able to:}
\begin{itemize}
\item Install Anaconda and launch Jupyter Notebook
\item Create and execute Python code cells
\item Understand and use basic data types (int, float, str, bool)
\item Store stock prices and financial data in variables
\end{itemize}

\vspace{1em}
\textbf{Finance Application:} Store and manipulate stock prices using Python variables.

\bottomnote{Foundation lesson -- everything builds on these basics}
\end{frame}

% Jupyter Interface
\begin{frame}[t]{The Jupyter Notebook Environment}
\begin{columns}[T]
\column{0.48\textwidth}
\textbf{Why Jupyter?}
\begin{itemize}
\item Interactive code execution
\item Mix code, output, and documentation
\item Industry standard for data science
\item Perfect for financial analysis
\end{itemize}

\vspace{0.5em}
\textbf{Getting Started:}
\begin{enumerate}
\item Install Anaconda from anaconda.com
\item Launch Jupyter Notebook
\item Create New $\rightarrow$ Python 3
\end{enumerate}

\column{0.48\textwidth}
\begin{center}
\includegraphics[width=\textwidth]{01_jupyter_interface/chart.pdf}
\end{center}
\end{columns}

\bottomnote{Jupyter = Julia + Python + R -- supports multiple languages}
\end{frame}

% Data Types
\begin{frame}[t]{Python Data Types}
\begin{columns}[T]
\column{0.48\textwidth}
\begin{center}
\includegraphics[width=\textwidth]{02_data_types_hierarchy/chart.pdf}
\end{center}

\column{0.48\textwidth}
\textbf{Four Basic Types:}

\vspace{0.3em}
\texttt{int} -- Integers (whole numbers)\\
\texttt{~~~price\_shares = 100}

\vspace{0.3em}
\texttt{float} -- Decimals\\
\texttt{~~~stock\_price = 185.50}

\vspace{0.3em}
\texttt{str} -- Text strings\\
\texttt{~~~ticker = "AAPL"}

\vspace{0.3em}
\texttt{bool} -- True/False\\
\texttt{~~~is\_profitable = True}
\end{columns}

\bottomnote{Use type(variable) to check any variable's type}
\end{frame}

% Variable Assignment
\begin{frame}[t]{Variable Assignment}
\begin{center}
\includegraphics[width=0.7\textwidth]{03_variable_assignment/chart.pdf}
\end{center}

\textbf{Key Rules:}
\begin{itemize}
\item Variables store values in memory
\item Names must start with letter or underscore
\item Case sensitive: \texttt{Price} $\neq$ \texttt{price}
\item Use descriptive names: \texttt{aapl\_price} not \texttt{x}
\end{itemize}

\bottomnote{Python uses dynamic typing -- no need to declare types}
\end{frame}

% Integer vs Float
\begin{frame}[t]{Integers vs Floats in Finance}
\begin{columns}[T]
\column{0.55\textwidth}
\begin{center}
\includegraphics[width=\textwidth]{04_int_vs_float/chart.pdf}
\end{center}

\column{0.42\textwidth}
\textbf{When to use integers:}
\begin{itemize}
\item Number of shares: \texttt{shares = 100}
\item Trading days: \texttt{days = 252}
\item Position count: \texttt{positions = 5}
\end{itemize}

\vspace{0.5em}
\textbf{When to use floats:}
\begin{itemize}
\item Stock prices: \texttt{price = 185.50}
\item Returns: \texttt{ret = 0.0523}
\item Percentages: \texttt{pct = 5.23}
\end{itemize}
\end{columns}

\bottomnote{Division always returns float: 10 / 3 = 3.333...}
\end{frame}

% String Operations
\begin{frame}[t]{String Operations for Finance}
\begin{columns}[T]
\column{0.48\textwidth}
\begin{center}
\includegraphics[width=\textwidth]{05_string_operations/chart.pdf}
\end{center}

\column{0.48\textwidth}
\textbf{Common String Operations:}

\vspace{0.3em}
\texttt{ticker = "AAPL"}\\
\texttt{ticker.upper()} $\rightarrow$ \texttt{"AAPL"}\\
\texttt{ticker.lower()} $\rightarrow$ \texttt{"aapl"}

\vspace{0.5em}
\textbf{String Formatting:}\\
\texttt{f"Price: \$\{price\}"}

\vspace{0.5em}
\textbf{Concatenation:}\\
\texttt{"NASDAQ:" + ticker}
\end{columns}

\bottomnote{F-strings (f"...") are the modern way to format strings}
\end{frame}

% Boolean Logic
\begin{frame}[t]{Boolean Logic for Trading Decisions}
\begin{columns}[T]
\column{0.55\textwidth}
\begin{center}
\includegraphics[width=\textwidth]{06_boolean_logic/chart.pdf}
\end{center}

\column{0.42\textwidth}
\textbf{Comparison Operators:}
\begin{itemize}
\item \texttt{>} greater than
\item \texttt{<} less than
\item \texttt{>=} greater or equal
\item \texttt{==} equal to
\item \texttt{!=} not equal
\end{itemize}

\vspace{0.3em}
\textbf{Example:}\\
\texttt{price > 200} $\rightarrow$ \texttt{True/False}
\end{columns}

\bottomnote{Booleans are essential for trading rule logic}
\end{frame}

% Type Conversion
\begin{frame}[t]{Type Conversion}
\begin{columns}[T]
\column{0.48\textwidth}
\begin{center}
\includegraphics[width=\textwidth]{07_type_conversion/chart.pdf}
\end{center}

\column{0.48\textwidth}
\textbf{Converting Between Types:}

\vspace{0.3em}
\texttt{int("100")} $\rightarrow$ \texttt{100}\\
\texttt{float("185.5")} $\rightarrow$ \texttt{185.5}\\
\texttt{str(185.5)} $\rightarrow$ \texttt{"185.5"}\\
\texttt{bool(1)} $\rightarrow$ \texttt{True}

\vspace{0.5em}
\textbf{Finance Use Case:}\\
Reading prices from CSV files\\
(data comes as strings)
\end{columns}

\bottomnote{Be careful: int("185.5") fails -- convert to float first}
\end{frame}

% Python vs Excel
\begin{frame}[t]{Why Python Instead of Excel?}
\begin{center}
\includegraphics[width=0.85\textwidth]{08_python_vs_excel/chart.pdf}
\end{center}

\bottomnote{Python scales to millions of rows; Excel struggles past 100K}
\end{frame}

% Hands-on Exercise
\begin{frame}[t]{Hands-on Exercise (25 min)}
\textbf{Create a Jupyter notebook and complete:}

\begin{enumerate}
\item Create variables for a stock portfolio:
\begin{itemize}
\item \texttt{ticker = "AAPL"} (string)
\item \texttt{shares = 50} (integer)
\item \texttt{buy\_price = 150.25} (float)
\item \texttt{current\_price = 185.50} (float)
\end{itemize}

\item Calculate portfolio metrics:
\begin{itemize}
\item Total investment: \texttt{shares * buy\_price}
\item Current value: \texttt{shares * current\_price}
\item Profit: current value - investment
\item Return \%: \texttt{(profit / investment) * 100}
\end{itemize}

\item Create a boolean: \texttt{is\_profitable = profit > 0}

\item Print results using f-strings
\end{enumerate}

\bottomnote{Save your notebook -- we'll build on this next lesson}
\end{frame}

% Summary
\begin{frame}[t]{Lesson Summary}
\textbf{Key Takeaways:}
\begin{itemize}
\item Jupyter Notebook is our development environment
\item Four basic types: \texttt{int}, \texttt{float}, \texttt{str}, \texttt{bool}
\item Variables store values with descriptive names
\item Type conversion needed when reading external data
\item Python handles large datasets better than Excel
\end{itemize}

\vspace{1em}
\textbf{Next Lesson:} Data Structures (Lists and Dictionaries)

\bottomnote{Practice: Experiment with different variable types in Jupyter}
\end{frame}

\end{document}
