\documentclass[8pt,aspectratio=169]{beamer}
\usetheme{Madrid}
\usepackage{graphicx}
\usepackage{booktabs}
\usepackage{amsmath}

\definecolor{mlpurple}{RGB}{51,51,178}
\definecolor{mllavender}{RGB}{173,173,224}
\definecolor{mllavender3}{RGB}{204,204,235}

\setbeamercolor{palette primary}{bg=mllavender3,fg=mlpurple}
\setbeamercolor{structure}{fg=mlpurple}
\setbeamercolor{frametitle}{fg=mlpurple,bg=mllavender3}

\setbeamertemplate{navigation symbols}{}
\setbeamersize{text margin left=5mm,text margin right=5mm}

\newcommand{\bottomnote}[1]{\vfill\footnotesize\textbf{#1}}


\title{Lesson 17: Matplotlib Basics}
\subtitle{Data Science with Python -- BSc Course}
\date{45 Minutes}

\begin{document}

\begin{frame}[plain]
\titlepage
\end{frame}

\begin{frame}[t]{Learning Objectives}
\textbf{After this lesson, you will be able to:}
\begin{itemize}
\item Create line, bar, and scatter plots
\item Customize colors, labels, legends
\item Build multi-panel figures
\item Add annotations and formatting
\end{itemize}
\bottomnote{Finance application: Statistical analysis of market data}
\end{frame}


\begin{frame}[t]{Basic Line Plot}
\begin{center}
\includegraphics[width=0.75\textwidth]{01a_basic_line/chart.pdf}
\end{center}
\bottomnote{plt.plot(x, y) for simple line charts}
\end{frame}


\begin{frame}[t]{Multiple Lines}
\begin{center}
\includegraphics[width=0.75\textwidth]{01b_multiple_lines/chart.pdf}
\end{center}
\bottomnote{Comparing multiple time series}
\end{frame}


\begin{frame}[t]{Line with Markers}
\begin{center}
\includegraphics[width=0.75\textwidth]{01c_line_markers/chart.pdf}
\end{center}
\bottomnote{Adding data point markers}
\end{frame}


\begin{frame}[t]{Fill Between}
\begin{center}
\includegraphics[width=0.75\textwidth]{01d_fill_between/chart.pdf}
\end{center}
\bottomnote{Shading area between lines}
\end{frame}


\begin{frame}[t]{Vertical Bar}
\begin{center}
\includegraphics[width=0.75\textwidth]{02a_vertical_bar/chart.pdf}
\end{center}
\bottomnote{Comparing categorical values}
\end{frame}


\begin{frame}[t]{Horizontal Bar}
\begin{center}
\includegraphics[width=0.75\textwidth]{02b_horizontal_bar/chart.pdf}
\end{center}
\bottomnote{Better for long category names}
\end{frame}


\begin{frame}[t]{Grouped Bar}
\begin{center}
\includegraphics[width=0.75\textwidth]{02c_grouped_bar/chart.pdf}
\end{center}
\bottomnote{Comparing multiple series by category}
\end{frame}


\begin{frame}[t]{Stacked Bar}
\begin{center}
\includegraphics[width=0.75\textwidth]{02d_stacked_bar/chart.pdf}
\end{center}
\bottomnote{Showing composition of totals}
\end{frame}


\begin{frame}[t]{Basic Histogram}
\begin{center}
\includegraphics[width=0.75\textwidth]{03a_basic_histogram/chart.pdf}
\end{center}
\bottomnote{Distribution with mean and median}
\end{frame}


\begin{frame}[t]{Density Histogram}
\begin{center}
\includegraphics[width=0.75\textwidth]{03b_density_histogram/chart.pdf}
\end{center}
\bottomnote{Normalized with PDF overlay}
\end{frame}


\begin{frame}[t]{Comparing Distributions}
\begin{center}
\includegraphics[width=0.75\textwidth]{03c_comparing_distributions/chart.pdf}
\end{center}
\bottomnote{Multiple overlapping histograms}
\end{frame}


\begin{frame}[t]{Cumulative Histogram}
\begin{center}
\includegraphics[width=0.75\textwidth]{03d_cumulative_histogram/chart.pdf}
\end{center}
\bottomnote{Frequency with CDF}
\end{frame}


\begin{frame}[t]{Basic Scatter}
\begin{center}
\includegraphics[width=0.75\textwidth]{04a_basic_scatter/chart.pdf}
\end{center}
\bottomnote{Simple X vs Y relationship}
\end{frame}


\begin{frame}[t]{Scatter with Regression}
\begin{center}
\includegraphics[width=0.75\textwidth]{04b_scatter_regression/chart.pdf}
\end{center}
\bottomnote{Adding trend line}
\end{frame}


\begin{frame}[t]{Multidimensional Scatter}
\begin{center}
\includegraphics[width=0.75\textwidth]{04c_multidimensional_scatter/chart.pdf}
\end{center}
\bottomnote{Size and color encoding}
\end{frame}


\begin{frame}[t]{Grouped Scatter}
\begin{center}
\includegraphics[width=0.75\textwidth]{04d_grouped_scatter/chart.pdf}
\end{center}
\bottomnote{Comparing categories}
\end{frame}


\begin{frame}[t]{add\_subplot (sin)}
\begin{center}
\includegraphics[width=0.75\textwidth]{05a_add_subplot_sin/chart.pdf}
\end{center}
\bottomnote{fig.add\_subplot(rows, cols, index)}
\end{frame}


\begin{frame}[t]{add\_subplot (cos)}
\begin{center}
\includegraphics[width=0.75\textwidth]{05b_add_subplot_cos/chart.pdf}
\end{center}
\bottomnote{Multiple axes in one figure}
\end{frame}


\begin{frame}[t]{Spanning Subplot}
\begin{center}
\includegraphics[width=0.75\textwidth]{05c_spanning_subplot/chart.pdf}
\end{center}
\bottomnote{Subplot spanning multiple positions}
\end{frame}


\begin{frame}[t]{Subplots Iteration}
\begin{center}
\includegraphics[width=0.75\textwidth]{05d_subplots_iteration/chart.pdf}
\end{center}
\bottomnote{plt.subplots(rows, cols) and indexing}
\end{frame}


\begin{frame}[t]{Line Styles}
\begin{center}
\includegraphics[width=0.75\textwidth]{06a_line_styles/chart.pdf}
\end{center}
\bottomnote{Solid, dashed, dotted, markers}
\end{frame}


\begin{frame}[t]{Transparency}
\begin{center}
\includegraphics[width=0.75\textwidth]{06b_transparency/chart.pdf}
\end{center}
\bottomnote{Using alpha for layering}
\end{frame}


\begin{frame}[t]{Axis Customization}
\begin{center}
\includegraphics[width=0.75\textwidth]{06c_axis_customization/chart.pdf}
\end{center}
\bottomnote{Limits, ticks, labels}
\end{frame}


\begin{frame}[t]{Annotations and Text}
\begin{center}
\includegraphics[width=0.75\textwidth]{06d_annotations_text/chart.pdf}
\end{center}
\bottomnote{Highlighting key points}
\end{frame}


\begin{frame}[t]{Arrow Styles}
\begin{center}
\includegraphics[width=0.75\textwidth]{07a_arrow_styles/chart.pdf}
\end{center}
\bottomnote{Different annotation arrows}
\end{frame}


\begin{frame}[t]{Text Boxes}
\begin{center}
\includegraphics[width=0.75\textwidth]{07b_text_boxes/chart.pdf}
\end{center}
\bottomnote{bbox styles for annotations}
\end{frame}


\begin{frame}[t]{Event Markers}
\begin{center}
\includegraphics[width=0.75\textwidth]{07c_event_markers/chart.pdf}
\end{center}
\bottomnote{Highlighting specific dates}
\end{frame}


\begin{frame}[t]{Statistics Box}
\begin{center}
\includegraphics[width=0.75\textwidth]{07d_statistics_box/chart.pdf}
\end{center}
\bottomnote{Key metrics annotation}
\end{frame}


\begin{frame}[t]{Price + Volume}
\begin{center}
\includegraphics[width=0.75\textwidth]{08a_price_volume/chart.pdf}
\end{center}
\bottomnote{Dual axis chart}
\end{frame}


\begin{frame}[t]{Drawdown}
\begin{center}
\includegraphics[width=0.75\textwidth]{08b_drawdown/chart.pdf}
\end{center}
\bottomnote{Visualizing losses from peak}
\end{frame}


\begin{frame}[t]{Rolling Sharpe}
\begin{center}
\includegraphics[width=0.75\textwidth]{08c_rolling_sharpe/chart.pdf}
\end{center}
\bottomnote{Performance over time}
\end{frame}


\begin{frame}[t]{Cumulative Returns}
\begin{center}
\includegraphics[width=0.75\textwidth]{08d_cumulative_returns/chart.pdf}
\end{center}
\bottomnote{Strategy comparison}
\end{frame}


\begin{frame}[t]{Lesson Summary}
\textbf{Key Takeaways:}
\begin{itemize}
\item Line plots for time series, bar charts for categories
\item Histograms show distributions, scatter plots show relationships
\item Subplots combine multiple views
\item Customization: colors, styles, annotations, text boxes
\item Finance applications: price/volume, drawdown, Sharpe, returns
\end{itemize}
\bottomnote{Statistics + Visualization = Data Science foundation}
\end{frame}

\end{document}
