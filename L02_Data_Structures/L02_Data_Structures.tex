\documentclass[8pt,aspectratio=169]{beamer}
\usetheme{Madrid}
\usepackage{graphicx}
\usepackage{booktabs}
\usepackage{adjustbox}
\usepackage{multicol}
\usepackage{amsmath}
\usepackage{amssymb}

% Color definitions
\definecolor{mlblue}{RGB}{0,102,204}
\definecolor{mlpurple}{RGB}{51,51,178}
\definecolor{mllavender}{RGB}{173,173,224}
\definecolor{mllavender2}{RGB}{193,193,232}
\definecolor{mllavender3}{RGB}{204,204,235}
\definecolor{mllavender4}{RGB}{214,214,239}
\definecolor{mlorange}{RGB}{255, 127, 14}
\definecolor{mlgreen}{RGB}{44, 160, 44}
\definecolor{mlred}{RGB}{214, 39, 40}

\setbeamercolor{palette primary}{bg=mllavender3,fg=mlpurple}
\setbeamercolor{palette secondary}{bg=mllavender2,fg=mlpurple}
\setbeamercolor{palette tertiary}{bg=mllavender,fg=white}
\setbeamercolor{palette quaternary}{bg=mlpurple,fg=white}
\setbeamercolor{structure}{fg=mlpurple}
\setbeamercolor{frametitle}{fg=mlpurple,bg=mllavender3}
\setbeamercolor{block title}{bg=mllavender2,fg=mlpurple}
\setbeamercolor{block body}{bg=mllavender4,fg=black}

\setbeamertemplate{navigation symbols}{}
\setbeamertemplate{itemize items}[circle]
\setbeamersize{text margin left=5mm,text margin right=5mm}

\newcommand{\bottomnote}[1]{%
\vfill\vspace{-2mm}
\textcolor{mllavender2}{\rule{\textwidth}{0.4pt}}
\vspace{1mm}\footnotesize\textbf{#1}}


\title{Lesson 02: Data Structures}
\subtitle{Data Science with Python -- BSc Course}
\author{Data Science Program}
\date{45 Minutes}

\begin{document}

\begin{frame}[plain]
\titlepage
\end{frame}

\begin{frame}[t]{Learning Objectives}
\textbf{After this lesson, you will be able to:}
\begin{itemize}
\item Create and manipulate Python lists
\item Access elements using indexing and slicing
\item Build dictionaries for key-value data storage
\item Apply list comprehensions for efficient data processing
\end{itemize}

\vspace{1em}
\textbf{Finance Application:} Store portfolio holdings as lists and dictionaries.

\bottomnote{Data structures are containers for organizing information}
\end{frame}

\begin{frame}[t]{List Indexing}
\begin{columns}[T]
\column{0.55\textwidth}
\begin{center}
\includegraphics[width=\textwidth]{01_list_indexing/chart.pdf}
\end{center}

\column{0.42\textwidth}
\textbf{Creating Lists:}\\
\texttt{prices = [185, 190, 188, 195]}

\vspace{0.5em}
\textbf{Accessing Elements:}\\
\texttt{prices[0]} $\rightarrow$ \texttt{185} (first)\\
\texttt{prices[-1]} $\rightarrow$ \texttt{195} (last)\\
\texttt{prices[1]} $\rightarrow$ \texttt{190} (second)

\vspace{0.5em}
\textbf{Remember:}\\
Python indexing starts at 0!
\end{columns}

\bottomnote{Negative indices count from the end: -1 is last element}
\end{frame}

\begin{frame}[t]{Slicing Notation}
\begin{columns}[T]
\column{0.48\textwidth}
\begin{center}
\includegraphics[width=\textwidth]{02_slicing_notation/chart.pdf}
\end{center}

\column{0.48\textwidth}
\textbf{Slice Syntax:} \texttt{list[start:end:step]}

\vspace{0.5em}
\texttt{prices = [185, 190, 188, 195, 182]}

\vspace{0.3em}
\texttt{prices[1:4]} $\rightarrow$ \texttt{[190, 188, 195]}\\
\texttt{prices[:3]} $\rightarrow$ \texttt{[185, 190, 188]}\\
\texttt{prices[2:]} $\rightarrow$ \texttt{[188, 195, 182]}\\
\texttt{prices[::2]} $\rightarrow$ \texttt{[185, 188, 182]}

\vspace{0.5em}
\textbf{Key:} End index is exclusive
\end{columns}

\bottomnote{Slicing creates a new list -- original unchanged}
\end{frame}

\begin{frame}[t]{Dictionary Structure}
\begin{columns}[T]
\column{0.48\textwidth}
\begin{center}
\includegraphics[width=\textwidth]{03_dictionary_structure/chart.pdf}
\end{center}

\column{0.48\textwidth}
\textbf{Key-Value Pairs:}

\vspace{0.3em}
\texttt{portfolio = \{}\\
\texttt{~~~~"AAPL": 50,}\\
\texttt{~~~~"MSFT": 30,}\\
\texttt{~~~~"GOOGL": 20}\\
\texttt{\}}

\vspace{0.5em}
\textbf{Access by Key:}\\
\texttt{portfolio["AAPL"]} $\rightarrow$ \texttt{50}

\vspace{0.5em}
\texttt{portfolio.keys()}\\
\texttt{portfolio.values()}
\end{columns}

\bottomnote{Dictionaries provide O(1) lookup -- very fast access}
\end{frame}

\begin{frame}[t]{Nested Data Structures}
\begin{center}
\includegraphics[width=0.7\textwidth]{04_nested_structures/chart.pdf}
\end{center}

\textbf{Real-World Portfolio Structure:}\\
\texttt{holdings = \{"AAPL": \{"shares": 50, "price": 185.5\}, ....\}}

\bottomnote{Nested structures model complex financial data}
\end{frame}

\begin{frame}[t]{List Methods}
\begin{columns}[T]
\column{0.55\textwidth}
\begin{center}
\includegraphics[width=\textwidth]{05_list_methods/chart.pdf}
\end{center}

\column{0.42\textwidth}
\textbf{Adding Elements:}\\
\texttt{prices.append(200)}\\
\texttt{prices.insert(0, 180)}

\vspace{0.5em}
\textbf{Removing:}\\
\texttt{prices.remove(188)}\\
\texttt{prices.pop()} -- removes last

\vspace{0.5em}
\textbf{Sorting:}\\
\texttt{prices.sort()}\\
\texttt{prices.reverse()}
\end{columns}

\bottomnote{Methods modify the list in-place (except sorted())}
\end{frame}

\begin{frame}[t]{Portfolio as Dictionary}
\begin{columns}[T]
\column{0.48\textwidth}
\begin{center}
\includegraphics[width=\textwidth]{06_portfolio_dict/chart.pdf}
\end{center}

\column{0.48\textwidth}
\textbf{Portfolio Dictionary:}

\texttt{prices = \{}\\
\texttt{~~~~"AAPL": 185.50,}\\
\texttt{~~~~"MSFT": 378.20,}\\
\texttt{~~~~"GOOGL": 141.80}\\
\texttt{\}}

\vspace{0.5em}
\textbf{Calculate Total:}\\
\texttt{total = sum(prices.values())}

\vspace{0.5em}
\textbf{Check Existence:}\\
\texttt{"AAPL" in prices} $\rightarrow$ \texttt{True}
\end{columns}

\bottomnote{Dictionaries are ideal for ticker-to-data mappings}
\end{frame}

\begin{frame}[t]{List Comprehension}
\begin{columns}[T]
\column{0.48\textwidth}
\begin{center}
\includegraphics[width=\textwidth]{07_list_comprehension/chart.pdf}
\end{center}

\column{0.48\textwidth}
\textbf{Traditional Loop:}\\
\texttt{returns = []}\\
\texttt{for p in prices:}\\
\texttt{~~~~returns.append(p * 1.05)}

\vspace{0.5em}
\textbf{List Comprehension:}\\
\texttt{returns = [p * 1.05 for p in prices]}

\vspace{0.5em}
\textbf{With Condition:}\\
\texttt{high = [p for p in prices if p > 190]}
\end{columns}

\bottomnote{Comprehensions are more Pythonic and often faster}
\end{frame}

\begin{frame}[t]{Choosing the Right Structure}
\begin{center}
\includegraphics[width=0.85\textwidth]{08_structure_selection/chart.pdf}
\end{center}

\bottomnote{Lists for sequences, dicts for lookups, sets for uniqueness}
\end{frame}

\begin{frame}[t]{Hands-on Exercise (25 min)}
\textbf{Build a portfolio tracker:}

\begin{enumerate}
\item Create a list of stock tickers:\\
\texttt{tickers = ["AAPL", "MSFT", "GOOGL", "AMZN"]}

\item Create a dictionary with shares owned:\\
\texttt{shares = \{"AAPL": 50, "MSFT": 30, ...\}}

\item Create a dictionary with current prices

\item Calculate portfolio value using list comprehension:\\
\texttt{values = [shares[t] * prices[t] for t in tickers]}

\item Find total portfolio value: \texttt{sum(values)}

\item Filter stocks worth more than \$5000
\end{enumerate}

\bottomnote{Save your work -- we'll add more features next lesson}
\end{frame}

\begin{frame}[t]{Lesson Summary}
\textbf{Key Takeaways:}
\begin{itemize}
\item Lists store ordered sequences (accessed by index)
\item Dictionaries store key-value pairs (accessed by key)
\item Slicing extracts portions: \texttt{list[start:end]}
\item List comprehensions create lists efficiently
\item Choose structure based on access pattern
\end{itemize}

\vspace{1em}
\textbf{Next Lesson:} Control Flow (if/else, loops)

\bottomnote{Data structures + control flow = programming logic}
\end{frame}

\end{document}
