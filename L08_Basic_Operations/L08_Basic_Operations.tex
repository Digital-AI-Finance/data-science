\documentclass[8pt,aspectratio=169]{beamer}
\usetheme{Madrid}
\usepackage{graphicx}
\usepackage{booktabs}
\usepackage{amsmath}

\definecolor{mlpurple}{RGB}{51,51,178}
\definecolor{mllavender}{RGB}{173,173,224}
\definecolor{mllavender2}{RGB}{193,193,232}
\definecolor{mllavender3}{RGB}{204,204,235}
\definecolor{mllavender4}{RGB}{214,214,239}
\definecolor{mlorange}{RGB}{255, 127, 14}
\definecolor{mlgreen}{RGB}{44, 160, 44}

\setbeamercolor{palette primary}{bg=mllavender3,fg=mlpurple}
\setbeamercolor{structure}{fg=mlpurple}
\setbeamercolor{frametitle}{fg=mlpurple,bg=mllavender3}

\setbeamertemplate{navigation symbols}{}
\setbeamersize{text margin left=5mm,text margin right=5mm}

\newcommand{\bottomnote}[1]{\vfill\footnotesize\textbf{#1}}


\title{Lesson 08: Basic Operations}
\subtitle{Data Science with Python -- BSc Course}
\date{45 Minutes}

\begin{document}

\begin{frame}[plain]
\titlepage
\end{frame}


\begin{frame}[t]{Learning Objectives}
\textbf{After this lesson, you will be able to:}
\begin{itemize}
\item Creating new columns
\item apply() for transformations
\item Arithmetic operations
\item Sorting with sort_values()
\item Calculating returns and moving averages
\end{itemize}
\bottomnote{Finance application: Stock data processing and analysis}
\end{frame}


\begin{frame}[t]{Column Creation}
\begin{center}
\includegraphics[width=0.85\textwidth]{01_column_creation/chart.pdf}
\end{center}
\bottomnote{Key concept for financial data analysis}
\end{frame}


\begin{frame}[t]{Apply Function}
\begin{center}
\includegraphics[width=0.85\textwidth]{02_apply_function/chart.pdf}
\end{center}
\bottomnote{Key concept for financial data analysis}
\end{frame}


\begin{frame}[t]{03 Arithmetic}
\begin{center}
\includegraphics[width=0.85\textwidth]{03_arithmetic/chart.pdf}
\end{center}
\bottomnote{Key concept for financial data analysis}
\end{frame}


\begin{frame}[t]{04 Sorting}
\begin{center}
\includegraphics[width=0.85\textwidth]{04_sorting/chart.pdf}
\end{center}
\bottomnote{Key concept for financial data analysis}
\end{frame}


\begin{frame}[t]{05 Value Counts}
\begin{center}
\includegraphics[width=0.85\textwidth]{05_value_counts/chart.pdf}
\end{center}
\bottomnote{Key concept for financial data analysis}
\end{frame}


\begin{frame}[t]{06 Returns}
\begin{center}
\includegraphics[width=0.85\textwidth]{06_returns/chart.pdf}
\end{center}
\bottomnote{Key concept for financial data analysis}
\end{frame}


\begin{frame}[t]{07 Moving Average}
\begin{center}
\includegraphics[width=0.85\textwidth]{07_moving_average/chart.pdf}
\end{center}
\bottomnote{Key concept for financial data analysis}
\end{frame}


\begin{frame}[t]{08 Cheat Sheet}
\begin{center}
\includegraphics[width=0.85\textwidth]{08_cheat_sheet/chart.pdf}
\end{center}
\bottomnote{Key concept for financial data analysis}
\end{frame}


\begin{frame}[t]{Lesson Summary}
\textbf{Key Takeaways:}
\begin{itemize}
\item Creating new columns
\item apply() for transformations
\item Arithmetic operations
\item Sorting with sort_values()
\item Calculating returns and moving averages
\end{itemize}

\vspace{1em}
\textbf{Practice:} Apply these concepts to the stock price dataset.
\end{frame}

\end{document}
