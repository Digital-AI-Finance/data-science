\documentclass[8pt,aspectratio=169]{beamer}
\usetheme{Madrid}
\usepackage{graphicx}
\usepackage{booktabs}
\usepackage{amsmath}

\definecolor{mlpurple}{RGB}{51,51,178}
\definecolor{mllavender}{RGB}{173,173,224}
\definecolor{mllavender3}{RGB}{204,204,235}

\setbeamercolor{palette primary}{bg=mllavender3,fg=mlpurple}
\setbeamercolor{structure}{fg=mlpurple}
\setbeamercolor{frametitle}{fg=mlpurple,bg=mllavender3}

\setbeamertemplate{navigation symbols}{}
\setbeamersize{text margin left=5mm,text margin right=5mm}

\newcommand{\bottomnote}[1]{\vfill\footnotesize\textbf{#1}}


\title{Lesson 21: Linear Regression}
\subtitle{Data Science with Python -- BSc Course}
\date{45 Minutes}

\begin{document}

\begin{frame}[plain]
\titlepage
\end{frame}

\begin{frame}[t]{Learning Objectives}
\textbf{The Problem:}
A portfolio manager needs to understand how stocks respond to market movements.
How do we quantify systematic risk?

\vspace{0.3em}
\textbf{After this lesson, you will be able to:}
\begin{itemize}
\item Understand OLS estimation and the least squares principle
\item Fit linear models using sklearn's LinearRegression
\item Interpret coefficients (slope as beta, intercept as alpha)
\item Estimate CAPM beta to classify stocks by risk profile
\end{itemize}
\bottomnote{Finance Application: Stock classification for portfolio construction}
\end{frame}


\begin{frame}[t]{Regression Concept}
\textbf{Finding the Best-Fit Line}
\begin{itemize}
\item Linear regression finds the line that best describes the relationship
\item In finance: How does stock return respond to market return?
\end{itemize}
\begin{center}
\includegraphics[width=0.55\textwidth]{01_regression_concept/chart.pdf}
\end{center}
\bottomnote{The ``best'' line minimizes the sum of squared vertical distances (residuals)}
\end{frame}


\begin{frame}[t]{Which Line Fits Best?}
\begin{center}
\includegraphics[width=0.65\textwidth]{01b_multiple_lines/chart.pdf}
\end{center}
\bottomnote{OLS finds the unique line that minimizes total squared error}
\end{frame}


\begin{frame}[t]{OLS Formula}
\textbf{Ordinary Least Squares: The Math}
\begin{itemize}
\item Goal: Find $\beta_0$, $\beta_1$ that minimize $\sum(y_i - \hat{y}_i)^2$
\item Solution: $\beta_1 = \frac{\sum(x_i - \bar{x})(y_i - \bar{y})}{\sum(x_i - \bar{x})^2}$
\end{itemize}
\begin{center}
\includegraphics[width=0.55\textwidth]{02_ols_formula/chart.pdf}
\end{center}
\bottomnote{Why squares? (1) Makes errors positive, (2) Penalizes large errors more}
\end{frame}


\begin{frame}[t]{sklearn API}
\textbf{Implementation in Python}
\begin{itemize}
\item \texttt{from sklearn.linear\_model import LinearRegression}
\item \texttt{model = LinearRegression().fit(X, y)}
\item Access: \texttt{model.coef\_} (slope), \texttt{model.intercept\_}
\end{itemize}
\begin{center}
\includegraphics[width=0.55\textwidth]{03_sklearn_api/chart.pdf}
\end{center}
\bottomnote{Pattern: model.fit(X, y) then model.predict(X\_new) -- works for all sklearn models}
\end{frame}


\begin{frame}[t]{Coefficient Interpretation}
\textbf{What Do the Numbers Mean?}
\begin{itemize}
\item \textbf{Slope ($\beta_1$):} For each 1\% market move, stock moves $\beta_1$\%
\item \textbf{Intercept ($\beta_0$):} Stock's return when market return is zero
\end{itemize}
\begin{center}
\includegraphics[width=0.55\textwidth]{04_coefficient_interpretation/chart.pdf}
\end{center}
\bottomnote{Finance translation: Slope = beta (systematic risk), Intercept = alpha (skill)}
\end{frame}


\begin{frame}[t]{Different Slopes = Different Betas}
\begin{center}
\includegraphics[width=0.65\textwidth]{04b_slopes_comparison/chart.pdf}
\end{center}
\bottomnote{Higher beta = stock amplifies market moves more}
\end{frame}


\begin{frame}[t]{Making Predictions}
\textbf{Using the Model for Forecasting}
\begin{itemize}
\item Once fitted, predict stock return for any market scenario
\item \texttt{predicted = model.predict([[market\_return]])}
\end{itemize}
\begin{center}
\includegraphics[width=0.55\textwidth]{05_prediction_line/chart.pdf}
\end{center}
\bottomnote{Caution: Predictions assume the relationship stays stable}
\end{frame}


\begin{frame}[t]{Predicted vs Actual}
\begin{center}
\includegraphics[width=0.50\textwidth]{05b_predicted_vs_actual/chart.pdf}
\end{center}
\bottomnote{Points on the diagonal = perfect predictions. R-squared measures fit quality.}
\end{frame}


\begin{frame}[t]{Extrapolation Warning}
\begin{center}
\includegraphics[width=0.65\textwidth]{05c_extrapolation_warning/chart.pdf}
\end{center}
\bottomnote{Never predict outside your training data range!}
\end{frame}


\begin{frame}[t]{Residuals}
\textbf{Checking Prediction Quality}
\begin{itemize}
\item Residual = Actual - Predicted ($e_i = y_i - \hat{y}_i$)
\item Good model: residuals should be random (no pattern)
\end{itemize}
\begin{center}
\includegraphics[width=0.55\textwidth]{06_residuals/chart.pdf}
\end{center}
\bottomnote{Plot residuals vs predicted: if you see a pattern, the model is missing something}
\end{frame}


\begin{frame}[t]{Residual Distribution}
\begin{center}
\includegraphics[width=0.65\textwidth]{06b_residual_histogram/chart.pdf}
\end{center}
\bottomnote{Normality assumption: residuals should follow a bell curve centered at zero}
\end{frame}


\begin{frame}[t]{Assumptions}
\textbf{When Does Linear Regression Work?}
\begin{itemize}
\item \textbf{Linearity:} Relationship is actually linear (not curved)
\item \textbf{Homoscedasticity:} Variance of errors is constant
\item \textbf{Normality:} Residuals are normally distributed
\end{itemize}
\begin{center}
\includegraphics[width=0.50\textwidth]{07_assumptions/chart.pdf}
\end{center}
\bottomnote{Finance reality: Stock returns often violate these -- check residuals}
\end{frame}


\begin{frame}[t]{Homoscedasticity Check}
\begin{center}
\includegraphics[width=0.65\textwidth]{07b_homoscedasticity/chart.pdf}
\end{center}
\bottomnote{Funnel shape = heteroscedasticity. Fix: weighted least squares or log transform}
\end{frame}


\begin{frame}[t]{CAPM Beta}
\textbf{The Solution: Stock Classification by Systematic Risk}
\begin{itemize}
\item \textbf{Beta $>$ 1:} Aggressive stock -- amplifies market moves
\item \textbf{Beta $<$ 1:} Defensive stock -- dampens volatility
\end{itemize}
\begin{center}
\includegraphics[width=0.55\textwidth]{08_capm_beta/chart.pdf}
\end{center}
\bottomnote{Alpha ($\beta_0$): Outperformance after risk adjustment}
\end{frame}


\begin{frame}[t]{Beta Comparison}
\begin{center}
\includegraphics[width=0.65\textwidth]{08b_beta_comparison/chart.pdf}
\end{center}
\bottomnote{Mix defensive (low beta) and aggressive (high beta) stocks based on risk tolerance}
\end{frame}


\begin{frame}[t]{Hands-On Exercise (25 min)}
\textbf{Task: Estimate Beta for Your Favorite Stock}
\begin{enumerate}
\item Download 1 year of daily returns for a stock (e.g., MSFT) and SPY
\item Fit: \texttt{model.fit(spy\_returns, stock\_returns)}
\item Extract and interpret: What is the beta? What is the alpha?
\item Plot the regression line with actual data points
\end{enumerate}

\vspace{0.3em}
\textbf{Deliverable:} Scatter plot with regression line, annotated with beta value.
\bottomnote{Extension: Compare beta estimates using different time periods (1yr vs 5yr)}
\end{frame}


\begin{frame}[t]{Lesson Summary}
\textbf{Problem Solved:}
We can now quantify systematic risk using CAPM beta via linear regression.

\vspace{0.3em}
\textbf{Key Takeaways:}
\begin{itemize}
\item OLS finds the line that minimizes squared errors
\item sklearn: \texttt{LinearRegression().fit(X, y)} -- three lines of code
\item Slope = beta (market sensitivity), Intercept = alpha (skill)
\end{itemize}

\vspace{0.3em}
\textbf{Next Lesson:} Regularization (L22) -- what happens with too many features?
\bottomnote{Memory: Beta = slope of stock vs market regression. High beta = high volatility.}
\end{frame}

\end{document}
