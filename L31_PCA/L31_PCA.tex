\documentclass[8pt,aspectratio=169]{beamer}
\usetheme{Madrid}
\usepackage{graphicx}
\usepackage{booktabs}
\usepackage{amsmath}

\definecolor{mlpurple}{RGB}{51,51,178}
\definecolor{mllavender}{RGB}{173,173,224}
\definecolor{mllavender3}{RGB}{204,204,235}

\setbeamercolor{palette primary}{bg=mllavender3,fg=mlpurple}
\setbeamercolor{structure}{fg=mlpurple}
\setbeamercolor{frametitle}{fg=mlpurple,bg=mllavender3}

\setbeamertemplate{navigation symbols}{}
\setbeamersize{text margin left=5mm,text margin right=5mm}

\newcommand{\bottomnote}[1]{\vfill\footnotesize\textbf{#1}}


\title{Lesson 31: PCA Dimensionality Reduction}
\subtitle{Data Science with Python -- BSc Course}
\date{45 Minutes}

\begin{document}

\begin{frame}[plain]
\titlepage
\end{frame}

\begin{frame}[t]{Learning Objectives}
\textbf{After this lesson, you will be able to:}
\begin{itemize}
\item Understand principal components
\item Apply PCA with sklearn
\item Interpret explained variance
\item Reduce feature dimensions
\end{itemize}
\bottomnote{Building towards your final project}
\end{frame}


\begin{frame}[t]{Pca Concept}
\begin{center}
\includegraphics[width=0.85\textwidth]{01_pca_concept/chart.pdf}
\end{center}
\end{frame}


\begin{frame}[t]{Eigenvalues}
\begin{center}
\includegraphics[width=0.85\textwidth]{02_eigenvalues/chart.pdf}
\end{center}
\end{frame}


\begin{frame}[t]{Sklearn Pca}
\begin{center}
\includegraphics[width=0.85\textwidth]{03_sklearn_pca/chart.pdf}
\end{center}
\end{frame}


\begin{frame}[t]{Scree Plot}
\begin{center}
\includegraphics[width=0.85\textwidth]{04_scree_plot/chart.pdf}
\end{center}
\end{frame}


\begin{frame}[t]{Explained Variance}
\begin{center}
\includegraphics[width=0.85\textwidth]{05_explained_variance/chart.pdf}
\end{center}
\end{frame}


\begin{frame}[t]{Component Loadings}
\begin{center}
\includegraphics[width=0.85\textwidth]{06_component_loadings/chart.pdf}
\end{center}
\end{frame}


\begin{frame}[t]{Visualization}
\begin{center}
\includegraphics[width=0.85\textwidth]{07_visualization/chart.pdf}
\end{center}
\end{frame}


\begin{frame}[t]{Factor Extraction}
\begin{center}
\includegraphics[width=0.85\textwidth]{08_factor_extraction/chart.pdf}
\end{center}
\end{frame}


\begin{frame}[t]{Lesson Summary}
\textbf{Key Takeaways:}
\begin{itemize}
\item Understand principal components
\item Apply PCA with sklearn
\item Interpret explained variance
\item Reduce feature dimensions
\end{itemize}
\bottomnote{Apply these skills in your final project}
\end{frame}

\end{document}
