\documentclass[8pt,aspectratio=169]{beamer}
\usetheme{Madrid}
\usepackage{graphicx}
\usepackage{booktabs}
\usepackage{amsmath}

\definecolor{mlpurple}{RGB}{51,51,178}
\definecolor{mllavender}{RGB}{173,173,224}
\definecolor{mllavender3}{RGB}{204,204,235}

\setbeamercolor{palette primary}{bg=mllavender3,fg=mlpurple}
\setbeamercolor{structure}{fg=mlpurple}
\setbeamercolor{frametitle}{fg=mlpurple,bg=mllavender3}

\setbeamertemplate{navigation symbols}{}
\setbeamersize{text margin left=5mm,text margin right=5mm}

\newcommand{\bottomnote}[1]{\vfill\footnotesize\textbf{#1}}


\title{Lesson 24: Factor Models}
\subtitle{Data Science with Python -- BSc Course}
\date{45 Minutes}

\begin{document}

\begin{frame}[plain]
\titlepage
\end{frame}

\begin{frame}[t]{Learning Objectives}
\textbf{The Problem:}
CAPM uses only market beta, but stocks also respond to size, value, and momentum.
How do we capture multiple sources of systematic risk?

\vspace{0.3em}
\textbf{After this lesson, you will be able to:}
\begin{itemize}
\item Build multi-factor regression models
\item Understand Fama-French factors (SMB, HML)
\item Interpret factor loadings and alpha
\item Create complete ML pipelines with sklearn
\end{itemize}
\bottomnote{Finance Application: Decomposing returns into systematic factors for attribution}
\end{frame}


\begin{frame}[t]{Factor Concept}
\textbf{From Single to Multiple Risk Sources}
\begin{itemize}
\item CAPM: $R_i - R_f = \alpha + \beta_M(R_M - R_f)$ -- only market risk
\item Multi-factor: Add size, value, momentum as additional explanatory variables
\end{itemize}
\begin{center}
\includegraphics[width=0.55\textwidth]{01_factor_concept/chart.pdf}
\end{center}
\bottomnote{Key insight: Different stocks have different exposures to different risk factors}
\end{frame}


\begin{frame}[t]{Fama-French Factors}
\textbf{The Classic Three-Factor Model}
\begin{itemize}
\item \textbf{SMB} (Small Minus Big): Small caps outperform large caps historically
\item \textbf{HML} (High Minus Low): Value stocks outperform growth stocks
\end{itemize}
\begin{center}
\includegraphics[width=0.55\textwidth]{02_fama_french/chart.pdf}
\end{center}
\bottomnote{Nobel Prize (2013): Fama showed these factors explain returns better than CAPM alone}
\end{frame}


\begin{frame}[t]{Factor Loadings}
\textbf{How Much Exposure Does a Stock Have?}
\begin{itemize}
\item Each stock has different sensitivities to each factor
\item Loading = regression coefficient on that factor
\end{itemize}
\begin{center}
\includegraphics[width=0.55\textwidth]{03_factor_loadings/chart.pdf}
\end{center}
\bottomnote{Example: TSLA has high market beta but negative HML (growth stock, not value)}
\end{frame}


\begin{frame}[t]{Alpha and Beta Decomposition}
\textbf{Separating Skill from Risk Exposure}
\begin{itemize}
\item $\alpha$: Return not explained by factors (manager skill or mispricing)
\item Multi-factor alpha is ``purer'' than CAPM alpha
\end{itemize}
\begin{center}
\includegraphics[width=0.55\textwidth]{04_alpha_beta/chart.pdf}
\end{center}
\bottomnote{Industry standard: Report alpha after controlling for Fama-French factors}
\end{frame}


\begin{frame}[t]{Multi-Factor Regression}
\textbf{Implementation in Python}
\begin{itemize}
\item Same sklearn API: \texttt{LinearRegression().fit(X, y)} where X has multiple columns
\item Each coefficient is a factor loading
\end{itemize}
\begin{center}
\includegraphics[width=0.55\textwidth]{05_multi_factor/chart.pdf}
\end{center}
\bottomnote{X matrix columns: [Mkt-RF, SMB, HML] -- intercept is alpha}
\end{frame}


\begin{frame}[t]{sklearn Pipeline}
\textbf{Combining Preprocessing and Modeling}
\begin{itemize}
\item \texttt{Pipeline([('scaler', StandardScaler()), ('reg', Ridge())])}
\item Prevents data leakage: scaler fits only on training data
\end{itemize}
\begin{center}
\includegraphics[width=0.45\textwidth]{06_pipeline_sklearn/chart.pdf}
\end{center}
\bottomnote{Best practice: Always use pipelines for reproducible workflows}
\end{frame}


\begin{frame}[t]{Model Persistence}
\textbf{Saving and Loading Trained Models}
\begin{itemize}
\item \texttt{joblib.dump(model, 'model.pkl')} -- save to disk
\item \texttt{model = joblib.load('model.pkl')} -- reload for production
\end{itemize}
\begin{center}
\includegraphics[width=0.50\textwidth]{07_model_persistence/chart.pdf}
\end{center}
\bottomnote{Deployment: Train once, deploy saved model to production}
\end{frame}


\begin{frame}[t]{Portfolio Factor Attribution}
\textbf{Where Did Your Returns Come From?}
\begin{itemize}
\item Decompose portfolio returns into factor contributions
\item Shows whether performance came from market timing, factor bets, or alpha
\end{itemize}
\begin{center}
\includegraphics[width=0.55\textwidth]{08_portfolio_factors/chart.pdf}
\end{center}
\bottomnote{Risk management: Understand your factor exposures before they surprise you}
\end{frame}


\begin{frame}[t]{Hands-On Exercise (25 min)}
\textbf{Task: Build a Fama-French Factor Model}
\begin{enumerate}
\item Download Fama-French 3-factor data from Kenneth French's website
\item Merge with your stock returns (AAPL or similar)
\item Fit multi-factor regression: stock returns vs [Mkt-RF, SMB, HML]
\item Interpret: What are the factor loadings? Is there alpha?
\item Compare $R^2$ to single-factor CAPM model
\end{enumerate}

\vspace{0.3em}
\textbf{Deliverable:} Table of factor loadings + comparison of $R^2$ values.
\bottomnote{Extension: Add momentum (UMD) as a fourth factor -- does $R^2$ improve?}
\end{frame}


\begin{frame}[t]{Lesson Summary}
\textbf{Problem Solved:}
We can now decompose stock returns into multiple systematic factors, giving better risk attribution than CAPM alone.

\vspace{0.3em}
\textbf{Key Takeaways:}
\begin{itemize}
\item Fama-French: Market + SMB (size) + HML (value)
\item Factor loadings = regression coefficients on each factor
\item Alpha after factors = true outperformance
\item sklearn pipelines ensure reproducible workflows
\end{itemize}

\vspace{0.3em}
\textbf{Next Lesson:} Classification (L25) -- predicting categories instead of numbers
\bottomnote{Memory: SMB = Small Minus Big (size), HML = High Minus Low (value)}
\end{frame}

\end{document}
